\section[Proofs]{HyperLMNtal に関して行なった証明}

\begin{frame}{}
  HyperLMNtal を定義したは良いものの,\\
  これで適切なのかは使ってみないとよくわからない

  なので
  \begin{enumerate}
  \item
    (flat) LMNtal において定義されている規則が導出できること
    \begin{itemize}
    \item アルファ変換など
    \end{itemize}
  \item
    計算モデルとして重要な性質を満たすこと
    \begin{enumerate}
    \item 自由変数の集合が合同な式・プロセス間で変化しない
    \item 遷移によって新たな自由変数が生成されることがない
    \end{enumerate}
  \end{enumerate}
  を証明した
\end{frame}

\subsection{アルファ変換の導出}

\begin{frame}{}
  まずはアルファ変換を導出する
  \begin{itemize}
  \item 補題を2つ証明する必要がある
  \end{itemize}
\end{frame}

\begin{frame}{補題 1:$\nu X.P \equiv P \mbox{ where } X \notin \mathit{fn}(P)$}

  \[\begin{array}{llr}
  & \nu X.P &
  \\
  \equiv_{\mbox{\scriptsize E5}}
  & \nu X.(\zero, P)
  & \mbox{where } P \equiv_{\scriptsize \mbox{E1}} (\zero, P)
  \\
  \equiv_{\mbox{\scriptsize E10}}
  & (\nu X.\zero, P)
  & \mbox{where } X \notin \mathit{fn}(P)
  \\
  \equiv_{\mbox{\scriptsize E4}}
  & (\zero, P)
  & \mbox{where } \nu X.\zero \equiv_{\mbox{\scriptsize E8}} \zero
  \\
  \equiv_{\mbox{\scriptsize E1}} & P
  \end{array}\]    

\end{frame}


\begin{frame}{補題 2:$P[X/X]$ は $P$ と構文的に等しい}
  \scriptsize
  
  We prove this by structural induction on processes.

  \scalebox{0.8}{\parbox{1.25\textwidth}{
      
      \begin{itemize}
      \item[Case] $\zero$ :\\
        \begin{quote}
          \(\zero[X/X] \overset{def}{=} \zero\)
        \end{quote}
      \item[Case] \(p(X_1, \ldots, X_m)\) :\\
        \begin{quote}
          \( p(X_1, \ldots, X_m)[X/X] \overset{def}{=} p(X_1[X/X], \ldots, X_m[X/X])
          = p(X_1, \ldots, X_m)\) 

          \hspace{1em} since \( X_i[X/X] = X_i \) where \(
          X_i[X/X] \overset{def}{=}
          \left\{
          \begin{array}{ll}
            X   & \mbox{if } X_i = X \\
            X_i & \mbox{if } X_i \neq X
          \end{array}
          \right.
          \)
        \end{quote}
      \item[Case] \((P, Q)\) :\\
        \begin{quote}
          We have \(P[X/X] = P\) and \(Q[X/X] = Q\) by induction hypothesis. 

          Therefore,
          \((P, Q)[X/X] \overset{def}{=} (P[X/X], Q[X/X]) = (P, Q)\)
        \end{quote}
      \item[Case] \(\nu Y.P\) :\\
        \begin{quote}
          \(
          (\nu Y.P)[X/X] \overset{def}{=}
          \left\{
          \begin{array}{ll}
            \nu Y.P & \mbox{if } Y = X \vspace{1em}\\
            \nu Y.P[X/X] & \mbox{if } Y \neq X \\
            = \nu Y.P & \because \mbox{
              \begin{minipage}[t]{15em}
                \(P[X/X] = P\) \\ 
                by induction hypothesis
              \end{minipage}
            }
          \end{array}
          \right.
          \)
          
          Since \(Y \neq X \land Y = X\) could never happen,
          there is no chance for ``variable capturing'' and
          $\alpha$-conversion for its avoidance
          (the third option of the link substitution scheme for the link creation),
          which possibly makes the process not syntactically equivalent
          (\(\alpha\)-equivalent though), won't happen.
        \end{quote}
      \item[Case] \((P \vdash Q)\) :\\
        \begin{quote}
          \( (P \vdash Q)[X/X] \overset{def}{=} (P \vdash Q)\)
        \end{quote}
      \end{itemize}
  }}
\end{frame}

\begin{frame}{アルファ変換の導出}
  
  \[\nu X.P \equiv \nu Y.P[Y / X] \mbox{ where } Y \notin \mathit{fn}(P)\]
  を導出する

  二つの場合に場合分けして考える

  \begin{myframe}
    \small
    \begin{itemize}
    \item[Case] \(X \in \mathit{fn}(P)\) :\\
      %      \begin{itemize}
    \item
      自由変数が含まれている場合
    \item
      まずはこちらを先に導出する
      %      \end{itemize}
    \item[Case] \(X \notin \mathit{fn}(P)\) :\\
      %      \begin{itemize}
    \item
      自由変数が \emph{含まれていない} 場合
    \item
      自明に思えるが,Link substitution によるアルファ変換があるため\\
      先に証明したパターンの形に無理やり変形しないと導出できない{\scriptsize はず}
      %      \end{itemize}
    \end{itemize}
  \end{myframe}
  
\end{frame}

\begin{frame}{
    \(X \in \mathit{fn}(P)\) の場合
  }
  
  
  \(\begin{array}{ll}
  \nu X. \nu Y. (Y \bowtie X, (X \bowtie Y, P)) &
  \\
  \equivby{E5, E6}
  \nu X. \nu Y. (X \bowtie X, P)
  & \because P[X/Y] = P \mbox{ since } Y \notin \mathit{fn}(P)
  \\
  \equivby{E5, Lemma 1}
  \nu X.(X \bowtie X, P)
  & \because Y \notin \mathit{fn}((X \bowtie X, P))
  \\
  \equivby{E6}
  \nu X.P
  & \because P[X/X] = P \mbox{ by Lemma 2}
  \\[\medskipamount]
  \mbox{and}
  \\[\medskipamount]
  \nu X. \nu Y. (Y \bowtie X, (X \bowtie Y, P)) &
  \\
  \equivby{E2, E3, E5, E9}
  \nu Y. \nu X. (X \bowtie Y, (Y \bowtie X, P)) &
  \\
  \equivby{E5, E6}
  \nu Y. \nu X. (Y \bowtie Y, P[Y/X]) & 
  \\
  \equivby{E5, Lemma 1}
  \nu Y. (Y \bowtie Y, P[Y/X])
  & \because X \notin \mathit{fn}((Y \bowtie Y, P[Y/X]))
  \\
  \equivby{E6}
  \nu Y.P[Y/X]
  & \because (P[Y/X])[Y/Y] = P[Y/X] \\
  & \ \ \mbox{ by Lemma 2}
  \end{array}\)
\end{frame}



\begin{frame}{
    \(X \notin \mathit{fn}(P)\) の場合
  }

  In this case, we first forcibly include free link \(X\)\\
  in order to exploit the former proof. 

  \(\begin{array}{ll}
  \nu X. P
  \\
  \equivby{Lemma 1}
  P
  \\
  \equivby{E1}
  (\zero, P)
  \\
  \equivby{E4, E7}
  (\nu X. \nu X. X \bowtie X, P)
  \\
  \equivby{E4, Lemma 1}
  (\nu X. X \bowtie X, P)
  & \because X \notin \mathit{fn}(\nu X. X \bowtie X)
  \\
  \equivby{E10}
  \nu X. (X \bowtie X, P)
  & \because X \notin \mathit{fn}(P)
  \\
  \equivby{The former proof}
  \nu Y. (X \bowtie X, P)[Y/X]
  & \because X \in \mathit{fn}((X \bowtie X, P))
  \\
  =
  \nu Y. (Y \bowtie Y, P[Y/X])
  \\
  \equivby{E6}
  \nu Y. (P[Y/X])[Y/Y]
  & \because Y \in \mathit{fn}((Y \bowtie Y, P[Y/X]))
  \\
  \equivby{Lemma 2}
  \nu Y. P[Y/X]
  \end{array}\)
  
\end{frame}


\begin{frame}{アルファ変換の導出に際して}
  実は全ての構造合同規則を用いていた
  \begin{itemize}
    \thusitem
    \Emph{Admissible な規則がない} ことの直感的な根拠
  \end{itemize}

  \vspace{2em}
  
  $X \bowtie Y \equiv Y \bowtie X$ (すぐに導出できる)と合わせると\\
  (flat) LMNtal の操作的意味論に対応する規則は全て定義・導出できた

\end{frame}

\subsection{自由リンクの集合に関する性質の証明}

\begin{frame}{}
  計算モデルとして重要な性質に
  \begin{enumerate}
  \item 自由変数の集合が合同な式・プロセス間で変化しない
    \begin{itemize}
    \item
      ラムダ計算はもちろん満たす
    \item
      パイ計算では match があるとダメだけど\\
      それを除けば満たす
    \end{itemize}
  \item 遷移によって新たな自由変数が生成されることがない
    \begin{itemize}
    \item 局所変数を内部的に生成したりするのはもちろん構わない
    \end{itemize}
  \end{enumerate}
  ということがある(と思う)

  \vspace{2em}

  それぞれ,構造合同規則・遷移規則に帰納法を適用して証明する
\end{frame}

\begin{frame}[allowframebreaks]{
    $\mathit{fn}(P) = \mathit{fn}(Q) \mbox{ if } P \equiv Q$
  }
  \scriptsize  
  \begin{itemize}
  \item[Case] \((\zero, P) \equiv P\) :\\
    \begin{quote}
      \(\mathit{fn}((\zero, P))
      = \mathit{fn}(\zero) \cup \mathit{fn}(P)
      = \mathit{fn}(P)\)
    \end{quote}
  \item[Case] \((P, Q) \equiv (Q, P)\) :\\
    \begin{quote}
      \(
      \mathit{fn}((P, Q))
      = \mathit{fn}(P) \cup \mathit{fn}(Q)
      = \mathit{fn}(Q) \cup \mathit{fn}(P)
      = \mathit{fn}((Q, P)) 
      \)
    \end{quote}
  \item[Case] \((P, (Q, R)) \equiv ((P, Q), R)\) :\\
    \begin{quote}
      \(
      \mathit{fn}((P, (Q, R)))
      = \mathit{fn}(P) \cup \mathit{fn}(Q) \cup \mathit{fn}(R)
      = \mathit{fn}(((P, Q), R))
      \)
    \end{quote}
  \item[Case] \(P \equiv P' \Rightarrow (P, Q) \equiv (P', Q)\) :\\
    \begin{quote}
      We have \(\mathit{fn}(P) = \mathit{fn}(P')\) if \(P \equiv P'\)
      by induction hypothesis.

      Therefore,
      \(
      \mathit{fn}((P, Q))
      = \mathit{fn}(P) \cup \mathit{fn}(Q)
      = \mathit{fn}(P') \cup \mathit{fn}(Q)
      = \mathit{fn}((P', Q))
      \)
    \end{quote}
  \item[Case] \(P \equiv P' \Rightarrow \nu X.P \equiv \nu X.P'\) :\\
    \begin{quote}
      We have \(\mathit{fn}(P) = \mathit{fn}(P')\) if \(P \equiv P'\)
      by induction hypothesis.
      
      Therefore,
      \(
      \mathit{fn}(\nu X.P)
      = \mathit{fn}(P) \setminus \{X\}
      = \mathit{fn}(P') \setminus \{X\}
      = \mathit{fn}(\nu X.P')
      \)
    \end{quote}
  \item[Case] \(\nu X.(X \bowtie Y, P) \equiv \nu X.P[Y / X]\)
    where \(X \in \mathit{fn}(P) \lor Y \in \mathit{fn}(P)\) :\\
    \begin{quote}
      Since \(X \in \mathit{fn}(P) \lor Y \in \mathit{fn}(P)\),
      \(\mathit{fn}(P[Y/X])\) should be equivalent with
      \((\mathit{fn}(P) \setminus \{X\}) \cup \{Y\} \)
      .
      
      Therefore, \\
      \(\begin{array}{l}
      \mathit{fn}(\nu X.(X \bowtie Y, P))
      = (\mathit{fn}(P) \cup \{Y\}) \setminus \{X\} \\
      = ((\mathit{fn}(P) \setminus \{X\}) \cup \{Y\}) \setminus \{X\}
      = \mathit{fn}(P[Y/X]) \setminus \{X\} 
      = \mathit{fn}(\nu X.P[Y/X])
      \end{array}\)
    \end{quote}
  \item[Case] \(\nu X.\nu Y.X \bowtie Y \equiv \zero\) :\\
    \begin{quote}
      \(
      \mathit{fn}(\nu X.\nu Y.X \bowtie Y)
      = \mathit{fn}(X \bowtie Y) \setminus \{X, Y\}
      = \{X, Y\} \setminus \{X, Y\}
      = \emptyset
      = \mathit{fn}(\zero)
      \)
    \end{quote}
  \item[Case] \(\nu X.\zero \equiv \zero\) :\\
    \begin{quote}
      \(
      \mathit{fn}(\nu X.\zero)
      = \emptyset \setminus \{X\}
      = \emptyset
      = \mathit{fn}(\zero)
      \)
    \end{quote}
  \item[Case] \(\nu X.\nu Y.P \equiv \nu Y.\nu X.P\) :\\
    \begin{quote}
      \(
      \mathit{fn}(\nu X.\nu Y.P)
      = (\mathit{fn}(P) \setminus \{Y\}) \setminus \{X\}
      = (\mathit{fn}(P) \setminus \{X\}) \setminus \{Y\}
      = \mathit{fn}(\nu Y.\nu X.P)
      \)
    \end{quote}
  \item[Case] \(\nu X.(P,Q) \equiv (\nu X.P,Q)\)
    where \(X \notin \mathit{fn}(Q)\) :\\
    \begin{quote}
      \(\begin{array}{ll}
      \mathit{fn}(\nu X.(P,Q))
      = (\mathit{fn}(P) \cup \mathit{fn}(Q)) \setminus \{X\} &\\
      = (\mathit{fn}(P) \setminus \{X\}) \cup \mathit{fn}(Q)
      & \because X \notin \mathit{fn}(Q) \\
      = \mathit{fn}((\nu X.P,Q)) &\\
      \end{array}\)
    \end{quote}
  \end{itemize}  
\end{frame}


\begin{frame}[allowframebreaks]{
    $\mathit{fn}(P) \supseteq \mathit{fn}(Q) \mbox{ if } P \longrightarrow Q$
  }
  \scriptsize
  
  \begin{itemize}
  \item[Case] \(\dfrac{P \longrightarrow P'}{(P, Q) \longrightarrow  (P', Q)}\) :\\
    \begin{quote}
      We have \(\mathit{fn}(P) \supseteq \mathit{fn}(P')\) if \(P \longrightarrow P'\)
      by induction hypothesis.

      Therefore,
      \(
      \mathit{fn}((P, Q))
      = \mathit{fn}(P) \cup \mathit{fn}(Q)
      \supseteq \mathit{fn}(P') \cup \mathit{fn}(Q)
      = \mathit{fn}((P', Q))
      \)
    \end{quote}
  \item[Case]
    \(\dfrac{P \longrightarrow P'}{\nu X.P \longrightarrow  \nu X.P'}\) :\\
    \begin{quote}
      We have
      \(\mathit{fn}(P) \supseteq \mathit{fn}(P')\) where \(P \longrightarrow P'\)
      by induction hypothesis.

      Therefore,
      \(
      \mathit{fn}((P, Q))
      = \mathit{fn}(P) \setminus \{X\}
      \supseteq \mathit{fn}(P') \setminus \{X\}
      = \mathit{fn}((P', Q))
      \)
    \end{quote}
  \item[Case] \(\dfrac{P \longrightarrow P'}{\nu X.P \longrightarrow  \nu X.P'}\) :\\
    \begin{quote}
      We have
      \(\mathit{fn}(P) \supseteq \mathit{fn}(P')\) where \(P \longrightarrow P'\)
      by induction hypothesis.

      Therefore,
      \(
      \mathit{fn}(\nu X.P)
      = \mathit{fn}(P) \setminus \{X\}
      \supseteq \mathit{fn}(P') \setminus \{X\}
      = \mathit{fn}(\nu X.P')
      \)
    \end{quote}
  \item[Case] \(
    \dfrac{Q \equiv P \hspace{16pt} P \longrightarrow P' \hspace{16pt} P' \equiv Q'}
          {Q \longrightarrow Q'}
          \) :\\
          \begin{quote}
            We have
            \(\mathit{fn}(P) \supseteq \mathit{fn}(P')\) where \(P \longrightarrow P'\)
            by induction hypothesis and\\
            \(\mathit{fn}(Q) = \mathit{fn}(P)\) where \(Q \equiv P\) and
            \(\mathit{fn}(P') = \mathit{fn}(Q')\) where \(P' \equiv Q'\) 
            by the previous theorem
            
            Therefore,
            \(
            \mathit{fn}(Q)
            = \mathit{fn}(P)
            \supseteq \mathit{fn}(P')
            = \mathit{fn}(Q')
            \)
          \end{quote}
        \item[Case] \((P, (P \vdash Q)) \longrightarrow (Q, (P \vdash Q))\) :\\
          \begin{quote}
            We have \(\mathit{fn}(P) \supseteq \mathit{fn}(Q)\) 
            by the second syntactical condition of a rule.

            Therefore,
            \(
            \mathit{fn}((P, (P \vdash Q)))
            = \mathit{fn}(P)
            \supseteq \mathit{fn}(Q) 
            = \mathit{fn}((Q, (P \vdash Q)))
            \)
          \end{quote}
  \end{itemize}

\end{frame}


