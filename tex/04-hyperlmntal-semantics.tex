\section[HyperLMNtal Syntax/Semantics]{HyperLMNtal の抽象構文と操作的意味論}

\begin{frame}{}
  今回紹介するのは膜がなく・全てハイパーリンクな flat HyperLMNtal だが,\\
  現実装に即した膜あり通常リンクあり版も卒論\footcite{sano}には書いた
\end{frame}

\begin{frame}{識別子と予約名}
  \begin{itemize}
    \item $X$ は(ハイパー)リンク名を表す
    \item $p$ はアトム名を表す
      \begin{itemize}
      \item \fbox{$\bowtie$} (fusion) のみ予約名
      \end{itemize}  
  \end{itemize}
\end{frame}

\begin{frame}{flat HyperLMNtal の抽象構文}
  \begin{center}
    \renewcommand{\arraystretch}{1.1}
    \begin{tabular}{ lrclll } 
      (Process) & $P$&$::=$& \(\zero\) && Null \\
      &&$|$& \(p (X_1, \ldots ,X_m)\) & \(m \geq 0\) & Atom \\
      &&$|$& \((P, P)\) && Molecule \\
      &&$|$& \Emph{\(\nu X.P\)} && \Emph{Link creation} \\
      &&$|$& \((P \vdash P)\) && Rule \\
    \end{tabular}
  \end{center}
\end{frame}

\begin{frame}{Free names}
  The set of the \Emph{free link names} in a process \mfbox{P}
  is denoted as \mfbox{\mathit{fn}(P)}\\
  and is defined inductively as following:
  
  \[\begin{aligned}
  \mathit{fn}(\zero)               &= \emptyset \\
  \mathit{fn}(p(X_1, \ldots ,X_m)) &= \bigcup_{i=1}^m \{X_i\} \\
  \mathit{fn}((P, Q))              &= \mathit{fn}(P) \cup \mathit{fn}(Q) \\
  \mathit{fn}(\nu X.P)             &= \mathit{fn}(P) \setminus \{X\} \\
  \mathit{fn}((P \vdash Q))        &= \emptyset
  \end{aligned}
  \]

\end{frame}
  

\begin{frame}{flat HyperLMNtal の構文条件}

  A rule \mfbox{(P \vdash Q)} must satisfy the following conditions:

  \begin{enumerate}
  \item Rules must not appear in \mfbox{P}
  \item \Emph{\(\mathit{fn}(P) \supseteq \mathit{fn}(Q)\)}
  \end{enumerate}

  \vspace{2em}
  
  Intuitively, the latter condition indicates\\
  that we have to denote a \emph{new} hyperlink in a scope of a \(\nu\) (new) on RHS,\\
  which we believe follows a common sense.
    
\end{frame}


\begin{frame}{Link substitution}
  \footnotesize
  \mfbox{P[Y/X]} replaces all free occurrences of \mfbox{X} with \mfbox{Y}
  \begin{itemize}
  \item
    If a free occurrence of $X$ occurs in a location where $Y$ would not be free,\\
    \underline{$\alpha$-conversion may be required}
  \item    
    we use \mfbox{=} to denote syntactic equivalence
  \end{itemize}
  
  \begin{center}
    \begin{tabular}{l}
      \( \zero[Y/X] \overset{def}{=} \zero \) \vspace{0.5em}\\ 
      \( p(X_1, \ldots, X_m)[Z/Y] \overset{def}{=} p(X_1[Z/Y], \ldots, X_m[Z/Y]) \)
      \vspace{0.3em}\\
      \hspace{11em} where \(
      X_i[Z/Y] \overset{def}{=}
      \left\{
      \begin{array}{ll}
        Z   & \mbox{if } X_i = Y \\
        X_i & \mbox{if } X_i \neq Y
      \end{array}
      \right.
      \) \vspace{0.5em}\\ 
      \((P, Q)[Y/X] \overset{def}{=} (P[Y/X], Q[Y/X])\) \vspace{0.5em}\\
      \(
      (\nu X.P)[Z/Y] \overset{def}{=}
      \left\{
      \arraycolsep = 0pt
      \begin{array}{ll}
        \nu X.P & \mbox{\hspace{1em} if } X = Y \\
        \nu X.P[Z/Y] & \mbox{\hspace{1em} if } X \neq Y \land X \neq Z \\
        \nu W.(P[W/X])[Z/Y] & \mbox{\hspace{1em} if }
        X \neq Y \land X = Z
        \land W \notin \mathit{fn}(P) \land W \neq Z
      \end{array}
      \right.
      \) \vspace{0.5em}\\
      \( (P \vdash Q)[Y/X] \overset{def}{=} (P \vdash Q) \) \\
    \end{tabular}
  \end{center}
  
\end{frame}


\begin{frame}{flat HyperLMNtal の操作的意味論|構造合同規則}
  \begin{center}
    \footnotesize
    \renewcommand{\arraystretch}{1.5}
    \begin{tabular}{ lrcll } 
      (E1)  & $(\mathbf{0}, P)$         &$\equiv$&      $P$\\
      (E2)  & $(P, Q)$                  &$\equiv$&      $(Q, P)$ \\
      (E3)  & $(P, (Q, R))$             &$\equiv$&      $((P, Q), R)$ \\
      (E4)  & $P \equiv P'$             &$\Rightarrow$& $(P, Q) \equiv (P', Q)$ \\
      (E5)  & $P \equiv Q$              &$\Rightarrow$& $\nu X.P \equiv \nu X.Q$ \\
      (E6)  & $\nu X.(X \bowtie Y, P)$  &$\equiv$&      $\nu X.P[Y / X]$
      & where \(X \in \mathit{fn}(P) \lor Y \in \mathit{fn}(P)\)\\
      (E7)  & $\nu X.\nu Y.X \bowtie Y$ &$\equiv$&      $\zero$ \\
      (E8)  & $\nu X.\zero$             &$\equiv$&      $\zero$\\
      (E9)  & $\nu X.\nu Y.P$           &$\equiv$&      $\nu Y.\nu X.P$\\
      (E10) & $\nu X.(P,Q)$             &$\equiv$&      $(\nu X.P,Q)$
       & where \(X \notin \mathit{fn}(Q)\)\\
    \end{tabular}
  \end{center}
\end{frame}

\begin{frame}{LMNtal の操作的意味論|遷移規則}
  \begin{center}
    \renewcommand{\arraystretch}{1.8}
    \begin{tabular}{ lc } 
      (R1) & \(\dfrac{P \longrightarrow P'}{(P, Q) \longrightarrow  (P', Q)} \)
      \vspace{1em}\\
      (R2) & \(\dfrac{P \longrightarrow P'}{\nu X.P \longrightarrow  \nu X.P'} \)
      \vspace{1em} \\
      (R3) & \(\dfrac{Q \equiv P \hspace{16pt} P \longrightarrow P' \hspace{16pt} P' \equiv Q'}{Q \longrightarrow Q'} \)
      \vspace{0.5em} \\
      (R4) & \( (P, (P \vdash Q)) \longrightarrow (Q, (P \vdash Q)) \) \\
    \end{tabular}
  \end{center}
\end{frame}


\begin{frame}{例題|Non-injective matching $\angled{1/2}$}
  Can the rule 
  \[
  (p(X, Y) \vdash q(X, Y))
  \]
  rewrite an atom \mfbox{p(X, X)} ?

  More precisely, can the process 
  \[
  (p(X, X), (p(X, Y) \vdash q(X, Y)))
  \]
  reduces to something?

  The rule cannot be $\alpha$-converted to the form
  \[ (p(X, X) \vdash \dots) \]

  \dots
\end{frame}
  
\begin{frame}{例題|Non-injective matching $\angled{2/2}$}
  However, the atom \mfbox{p(X, X)} can be converted to\\
  \mfbox{\nu Y.(Y \bowtie X, p(X, Y))} using (E7) and Lemma 1.

  Therefore, it can be rewritten as
  \begin{align*}
    & (p(X, X), (p(X, Y) \vdash q(X, Y))) \\
    & \equivby{Lemma 1}
    \nu Y.(p(X, X), (p(X, Y) \vdash q(X, Y)))\\
    & \equivby{E7}
    \nu Y.(Y \bowtie X, (p(X, Y), (p(X, Y) \vdash q(X, Y))))\\
    & \longrightarrow
    \nu Y.(Y \bowtie X, (q(X, Y), (p(X, Y) \vdash q(X, Y))))\\
    & \equivby{E7}
    \nu Y.(q(X, X), (p(X, Y) \vdash q(X, Y)))\\
    & \equivby{Lemma 1}
    (q(X, X), (p(X, Y) \vdash q(X, Y)))
  \end{align*}

  As the above, we can match non-injective free links\\
  using congruence rule on the link fusion. 
\end{frame}


